\documentclass[12pt, a4paper]{article}
\usepackage{bm,float,amsmath,graphicx,algorithm,algpseudocodex}
\graphicspath{ {./images/} }
\usepackage[T1]{fontenc}
\usepackage[polish]{babel}
\usepackage[utf8]{inputenc}

\title{Metody optymalizacji}
\author{Stepan Yurtsiv, 246437}
\date{9 maja 2022r.}

\begin{document}
\maketitle

\section*{Zadanie}

Celem niniejszego zadania jest zaimplementowanie algorytmu
aproksymacyjnego opartego na programowaniu liniowym dla
uogólnionego zagadnienia przydziału (ang. the generalized assignment problem)

\section*{Algorytm}

\textbf{Oznaczenia:}
\begin{itemize}
  \item $M = [m]$ - zbiór maszyn
  \item $J = [n]$ - zbiór zadań
  \item $T = \{T_i: i \in M\}$ - zbiór ograniczeń dla maszyn, $T_i$ - maksymalny czas działania $i$-tej maszyny
  \item $c = \{c_{ij}: i \in M, j \in J\}$ - macierz kosztów, $c_{ij}$ - koszt wykonania $j$-tego zadania na $i$-tej maszynie
  \item $p = \{p_{ij}: i \in M, j \in J\}$ - macierz czasów wykonania zadań, $p_{ij}$ - czas wykonania $j$-tego zadania na $i$-tej maszynie
\end{itemize}

Uogólnione zagadnienie przedziału polega na przypisaniu zbiorowi maszyn
$M$ zadań ze zbioru $J$ tak, żeby zmaksymalizować zyski z wykonania wszystkich
zadań. Dodatkowo są ograniczenia w postaci maksymalengo czasu przez który może
działać dana maszyna.

Rozwiązanie problemu sprowadza się do stworzenia grafu dwudzielnego
$G$, gdzie jedna grupa wierszchołków reprezentuje zbiór maszyn $M$, a inna
zbiór zadań $J$. Na początku mamy doczynienia z grafem pełnym, gdzie krawędź $(i, j)$ reprezentuje
przydział $i$-tej maszyny do $j$-tego zadania. Zbiór krawędzi
grafu $G$ oznaczamy przez $E$. W poszczególnych iteracjach
algorytmu \ref{alg} jest tworzony podgraf grafu $G$ oznaczany jako $F$, w którym
każdemu zadaniu jest przypisana dokładnie jedna maszyna.

Do rozwiązania podproblemu w każdej iteracji zastosujemy model (oznaczany jako $LP_{ga}$) z następującą 
funkcją celu:

\begin{center}
	max \textbf{$\displaystyle\sum_{e=(i,j) \in E} c_{ij} \cdot x_{ij}$}
\end{center}
gdzie $x_{ij}$ oznacza, czy $j$-te zadanie zostało przypisane do wykonania
na $i$-tej maszynie.
Ograniczenia do zmiennych decyzyjnych są następujące:

\begin{itemize}
  \item $\forall{j \in J} \displaystyle\sum_{e \in \delta(j)} x_e = 1$ - każde zadanie musi być przypisane dokładnie jednej maszynie
  \item $\forall{i \in M} \displaystyle\sum_{e \in \delta(i)} p_e \cdot x_e \leq T_i$ - czas wykonaia wszystkich zadań, przypisanych maszynie $i$ nie może
przekroczyć jej maksymalny czas dostępności
  \item $x_{ij} \geq 0$ - każda zmienna jest nieujmena
\end{itemize}

Algorytm \ref{alg} zapewnia, że każda maszyna jest używana
nie więcej niż dwukrotność jej dozwolonej dostępności.

\begin{algorithm}
\caption{}\label{alg}
\begin{enumerate}
  \item Inicjalizacja $E(F) \gets \emptyset$, $M' \gets M$ 
  \item While $J \neq \emptyset$ do
  \begin{enumerate}
    \item Znajdź optymalne rozwiązanie dopuszczalne bazowe $x$ do $LP_{ga}$
    i usuń wszystkie zmienne $x_{ij} = 0$
    \item Jeśli zmienna $x_{ij} = 1$, to zaktualizuj $F \gets F \cup \{(i, j)\}$
    , $J \gets J \setminus \{j\}$, $T_i \gets T_i - p_{ij}$
    \item (\textbf{Relaksacja}) Jeśli maszyna $i$ ma $d(i) = 1$ lub maszyna
    $i$ ma $d(i) = 2$ oraz $\displaystyle\sum_{j \in J} x_{ij} \geq 1$, to zaktualizuj $M' \gets M \setminus \{i\}$
  \end{enumerate}
  \item return $F$
\end{enumerate}
\end{algorithm}

\end{document}

