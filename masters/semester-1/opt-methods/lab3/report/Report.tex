\documentclass[12pt, a4paper]{article}
\usepackage{bm,float,amsmath,graphicx,algorithm,algpseudocodex,longtable}
\graphicspath{ {./images/} }
\usepackage[T1]{fontenc}
\usepackage[polish]{babel}
\usepackage[utf8]{inputenc}

\title{Metody optymalizacji}
\author{Stepan Yurtsiv, 246437}
\date{5 czerwca 2022r.}

\begin{document}
\maketitle

\section*{Zadanie}

Celem niniejszego zadania jest zaimplementowanie algorytmu
aproksymacyjnego opartego na programowaniu liniowym dla
uogólnionego zagadnienia przydziału (ang. the generalized assignment problem)

\section*{Algorytm}

\textbf{Oznaczenia:}
\begin{itemize}
  \item $M = [m]$ - zbiór maszyn
  \item $J = [n]$ - zbiór zadań
  \item $T = \{T_i: i \in M\}$ - zbiór ograniczeń dla maszyn, $T_i$ - maksymalny czas pracy $i$-tej maszyny
  \item $c = \{c_{ij}: i \in M, j \in J\}$ - macierz kosztów, $c_{ij}$ - koszt wykonania $j$-tego zadania na $i$-tej maszynie
  \item $p = \{p_{ij}: i \in M, j \in J\}$ - macierz czasów wykonania zadań, $p_{ij}$ - czas wykonania $j$-tego zadania na $i$-tej maszynie
\end{itemize}

Uogólnione zagadnienie przedziału polega na przypisaniu zbiorowi maszyn
$M$ zadań ze zbioru $J$ tak, żeby zminimalizować koszt z wykonania wszystkich
zadań. Dodatkowo są ograniczenia w postaci maksymalengo czasu przez który może
działać dana maszyna.

Rozwiązanie problemu sprowadza się do stworzenia grafu dwudzielnego
$G$, gdzie jedna grupa wierszchołków reprezentuje zbiór maszyn $M$, a inna
zbiór zadań $J$. Na początku mamy doczynienia z grafem pełnym, gdzie krawędź $(i, j)$ reprezentuje
przydział $i$-tej maszyny do $j$-tego zadania. Zbiór krawędzi
grafu $G$ oznaczamy przez $E$. W poszczególnych iteracjach
algorytmu \ref{alg} jest tworzony podgraf grafu $G$ oznaczany jako $F$, w którym
każdemu zadaniu jest przypisana dokładnie jedna maszyna.

Do rozwiązania podproblemu w każdej iteracji zastosujemy model (oznaczany jako $LP_{ga}$) z następującą 
funkcją celu:

\begin{center}
	min \textbf{$\displaystyle\sum_{e=(i,j) \in E} c_{ij} \cdot x_{ij}$}
\end{center}
gdzie $x_{ij}$ oznacza, czy $j$-te zadanie zostało przypisane do wykonania
na $i$-tej maszynie.
Ograniczenia do zmiennych decyzyjnych są następujące:

\begin{itemize}
  \item $\forall{j \in J} \displaystyle\sum_{e \in \delta(j)} x_e = 1$ - każde zadanie musi być przypisane dokładnie jednej maszynie
  \item $\forall{i \in M} \displaystyle\sum_{e \in \delta(i)} p_e \cdot x_e \leq T_i$ - czas wykonaia wszystkich zadań, przypisanych maszynie $i$ nie może
przekroczyć jej maksymalnego czasu dostępności
  \item $x_{ij} \geq 0$ - każda zmienna jest nieujmena
\end{itemize}

Algorytm \ref{alg} zapewnia, że każda maszyna jest używana
nie więcej niż dwukrotność jej dozwolonej dostępności.

\begin{algorithm}
\caption{}\label{alg}
\begin{enumerate}
  \item Inicjalizacja $E(F) \gets \emptyset$, $M' \gets M$ 
  \item While $J \neq \emptyset$ do
  \begin{enumerate}
    \item Znajdź optymalne rozwiązanie dopuszczalne bazowe $x$ do $LP_{ga}$
    i usuń wszystkie zmienne $x_{ij} = 0$
    \item Jeśli zmienna $x_{ij} = 1$, to zaktualizuj $F \gets F \cup \{(i, j)\}$
    , $J \gets J \setminus \{j\}$, $T_i \gets T_i - p_{ij}$
    \item (\textbf{Relaksacja}) Jeśli maszyna $i$ ma $d(i) = 1$ lub maszyna
    $i$ ma $d(i) = 2$ oraz $\displaystyle\sum_{j \in J} x_{ij} \geq 1$, to zaktualizuj $M' \gets M \setminus \{i\}$
  \end{enumerate}
  \item return $F$
\end{enumerate}
\end{algorithm}

\section*{Wyniki}

\textbf{Oznaczenia:}

\begin{itemize}
  \item $T_{max}$ - maksymalny dozwolny czas pracy wszystkich maszyn
  \item $T$ - czas pracy wszystkich maszyn w wyniku algorytmu 
\end{itemize}

\begin{longtable}{|c|c|c|c|c|c|c|}
  \hline
  Plik & Problem & Czas wykonania [ms] & Liczba iteracji & $T_{max}$ & $T$ & $T \mathbin{/} T_{max}$ \\
  \hline
  gap1 & 1 & 1.68 & 5 & 168 & 188 & 1.12 \\
  \hline
  gap1 & 2 & 1.52 & 6 & 203 & 208 & 1.02 \\
  \hline
  gap1 & 3 & 0.99 & 4 & 188 & 190 & 1.01 \\
  \hline
  gap1 & 4 & 1.17 & 5 & 187 & 201 & 1.07 \\
  \hline
  gap1 & 5 & 1.18 & 5 & 185 & 213 & 1.15 \\
  \hline
  gap2 & 1 & 1.88 & 7 & 250 & 268 & 1.07 \\
  \hline
  gap2 & 2 & 1.43 & 5 & 238 & 247 & 1.04 \\
  \hline
  gap2 & 3 & 1.73 & 6 & 235 & 245 & 1.04 \\
  \hline
  gap2 & 4 & 1.46 & 5 & 255 & 293 & 1.15 \\
  \hline
  gap2 & 5 & 1.44 & 5 & 231 & 245 & 1.06 \\
  \hline
  gap3 & 1 & 1.71 & 5 & 302 & 310 & 1.03 \\
  \hline
  gap3 & 2 & 2.27 & 7 & 268 & 295 & 1.1 \\
  \hline
  gap3 & 3 & 1.74 & 5 & 297 & 313 & 1.05 \\
  \hline
  gap3 & 4 & 1.74 & 5 & 305 & 312 & 1.02 \\
  \hline
  gap3 & 5 & 2.01 & 6 & 285 & 316 & 1.11 \\
  \hline
  gap4 & 1 & 2.32 & 6 & 355 & 360 & 1.01 \\
  \hline
  gap4 & 2 & 1.93 & 5 & 365 & 380 & 1.04 \\
  \hline
  gap4 & 3 & 1.94 & 5 & 367 & 389 & 1.06 \\
  \hline
  gap4 & 4 & 2.3 & 6 & 356 & 372 & 1.04 \\
  \hline
  gap4 & 5 & 2.74 & 7 & 349 & 388 & 1.11 \\
  \hline
  gap5 & 1 & 2.9 & 6 & 274 & 307 & 1.12 \\
  \hline
  gap5 & 2 & 2.51 & 5 & 280 & 285 & 1.02 \\
  \hline
  gap5 & 3 & 2.9 & 6 & 296 & 278 & 0.94 \\
  \hline
  gap5 & 4 & 2.65 & 5 & 273 & 280 & 1.03 \\
  \hline
  gap5 & 5 & 14.66 & 5 & 295 & 303 & 1.03 \\
  \hline
  gap6 & 1 & 2.95 & 5 & 379 & 393 & 1.04 \\
  \hline
  gap6 & 2 & 3.08 & 5 & 383 & 406 & 1.06 \\
  \hline
  gap6 & 3 & 2.97 & 5 & 368 & 392 & 1.07 \\
  \hline
  gap6 & 4 & 3.35 & 6 & 371 & 402 & 1.08 \\
  \hline
  gap6 & 5 & 3.35 & 6 & 390 & 425 & 1.09 \\
  \hline
  gap7 & 1 & 3.6 & 5 & 467 & 488 & 1.04 \\
  \hline
  gap7 & 2 & 5.18 & 6 & 466 & 495 & 1.06 \\
  \hline
  gap7 & 3 & 4.46 & 6 & 479 & 493 & 1.03 \\
  \hline
  gap7 & 4 & 4.21 & 6 & 495 & 508 & 1.03 \\
  \hline
  gap7 & 5 & 4.38 & 6 & 470 & 485 & 1.03 \\
  \hline
  gap8 & 1 & 4.87 & 5 & 403 & 411 & 1.02 \\
  \hline
  gap8 & 2 & 16.96 & 6 & 397 & 404 & 1.02 \\
  \hline
  gap8 & 3 & 5.09 & 6 & 403 & 414 & 1.03 \\
  \hline
  gap8 & 4 & 5.55 & 7 & 387 & 391 & 1.01 \\
  \hline
  gap8 & 5 & 5.03 & 6 & 405 & 424 & 1.05 \\
  \hline
  gap9 & 1 & 3.92 & 6 & 358 & 357 & 1.0 \\
  \hline
  gap9 & 2 & 4.23 & 7 & 359 & 378 & 1.05 \\
  \hline
  gap9 & 3 & 4.29 & 5 & 362 & 395 & 1.09 \\
  \hline
  gap9 & 4 & 4.19 & 7 & 361 & 374 & 1.04 \\
  \hline
  gap9 & 5 & 3.92 & 6 & 366 & 404 & 1.1 \\
  \hline
  gap10 & 1 & 5.0 & 6 & 471 & 476 & 1.01 \\
  \hline
  gap10 & 2 & 14.32 & 5 & 468 & 494 & 1.06 \\
  \hline
  gap10 & 3 & 5.77 & 6 & 487 & 509 & 1.05 \\
  \hline
  gap10 & 4 & 4.94 & 6 & 466 & 477 & 1.02 \\
  \hline
  gap10 & 5 & 5.4 & 7 & 466 & 468 & 1.0 \\
  \hline
  gap11 & 1 & 6.39 & 6 & 639 & 665 & 1.04 \\
  \hline
  gap11 & 2 & 7.35 & 7 & 646 & 645 & 1.0 \\
  \hline
  gap11 & 3 & 6.94 & 6 & 657 & 665 & 1.01 \\
  \hline
  gap11 & 4 & 7.42 & 7 & 641 & 698 & 1.09 \\
  \hline
  gap11 & 5 & 13.65 & 6 & 642 & 623 & 0.97 \\
  \hline
  gap12 & 1 & 8.47 & 6 & 721 & 752 & 1.04 \\
  \hline
  gap12 & 2 & 5.9 & 4 & 720 & 733 & 1.02 \\
  \hline
  gap12 & 3 & 6.91 & 5 & 719 & 761 & 1.06 \\
  \hline
  gap12 & 4 & 6.77 & 5 & 720 & 744 & 1.03 \\
  \hline
  gap12 & 5 & 7.04 & 5 & 708 & 769 & 1.09 \\
  \hline
\end{longtable}

\end{document}

